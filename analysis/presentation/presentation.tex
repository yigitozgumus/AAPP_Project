\documentclass{beamer}
\mode<presentation>{\usetheme{Warsaw}}
\usepackage[english]{babel}
\usepackage[utf8x]{inputenc}
\usepackage[T1]{fontenc}
\usepackage{hyperref}

\title{Advanced Algorithms and Parallel Programming Spring 2018}
\subtitle{Advanced Algorithms Project}
\author{Semsi Yigit Ozgumus}
\institute{Politecnico di Milano}
\date{\today}

\AtBeginSection[]
{
	\begin{frame}
	\frametitle{Current Section}
	\tableofcontents[currentsection]
	\end{frame}
}

\setbeamertemplate{sidebar right}{}
\setbeamertemplate{footline}{%
\hfill\usebeamertemplate***{navigation symbols}
\hspace{1cm}\insertframenumber{}/\inserttotalframenumber}

\begin{document}
	\begin{frame}
		\titlepage
	\end{frame}

	\begin{frame}
		\frametitle{Table of Contents}
		\tableofcontents
	\end{frame}
	
	\section{Introduction of the Problem}
		\begin{frame}
			\frametitle{What is a Strongly Connected Component ?}
			
			\begin{itemize}
				\item<2-> if a Vertex is reachable by any vertex in the graph, then that vertex is \textbf{strongly connected}.
				\item <3-> if a partition of vertices that are a subgraph of the graph are all strongly connected, then that subgraph is a \textbf{strongly connected component} of that graph.
				\item <4-> 	\begin{figure}[h!]
					\centering
					\includegraphics[width=.6\textwidth]{scc}
					\caption{Graph with strongly connected components}
				\end{figure}
			\end{itemize}
		\end{frame}
		
		\begin{frame}
			\frametitle{Methods to solve}
			\begin{itemize}
				\item <2-> Tarjan's Algorithm
				 \begin{itemize}
					\item <3-> Time Complexity $\rightarrow O(| V | + | E |)$
					\item <4-> Space Complexity $\rightarrow O(|V| \cdot (2 + 5 w)) $
				\end{itemize}
				\item <5-> Nuutila's Version
					\begin{itemize}
						\item <6-> Space Complexity  $\rightarrow O(|V| \cdot (1 + 4 w)) $
					\end{itemize}
				\item <7-> Pearce's Version
					\begin{itemize}
						\item <8-> Space Complexity  $\rightarrow O(|V| \cdot (1 + 3 w)) $
					\end{itemize}
			\end{itemize}
		\end{frame}
	
	\section{Implementation of the Project}
		\begin{frame}
			\frametitle{Project Structure}
			\begin{itemize}
				\item <1-> In terms of functionality the project has 4 different functions.
				\begin{enumerate}
					\item <2-> Generate a Random Directed Graph with options (Python)
					\item <3-> Run Experiement with algoritmhs while keeping track of completion time and storage consumption (tarjan vs. Nuutila vs. Pearce)
					\item <4-> Debug mode for testing a singular graph with algorithms provided.
					\item <5-> Visualization of the experiments, can only be done with cvs files (Python).
				\end{enumerate}
			\end{itemize}
		\end{frame}
		\begin{frame}
			\frametitle{Project Structure}
			\begin{itemize}
				\item <1-> All the steps are connected to each other and glued together to create a pipeline to emulate terminal application.
				\item <2-> All documentation done in Doxygen
				\item <3-> Visualization of the experiment is done with python, csv files can be further explored using the provided jupyter notebook.
			\end{itemize}
		\end{frame}
		\begin{frame}
			\frametitle{Python Part}
			\begin{itemize}
				\item <1-> There are 2 scripts that are embedded into the project.
				\begin{enumerate}
					\item <1-> generate_graph_directories.py
				\end{enumerate} 
			\end{itemize}
		\end{frame}
		\begin{frame}
			\frametitle{Project Structure}
			\begin{itemize}
				\item <1-6> Application Class uses:
				\begin{itemize}
					\item <2-6> Analyzer, Visualize and Session Classes
				\end{itemize}
				\item <3-6> Analyzer Class uses:
				\begin{itemize}
					\item <4-6> Nuutila, Pearce, Tarjan, Timer and GraphComponent Classes
				\end{itemize}
				\item <5-6> StorageItems struct is used by all algorithm classes to transfer the execution data 
			\end{itemize}
			\only<6>{	
			\begin{figure}[h!]
				\centering
				\includegraphics[width=1\textwidth]{application}
				\caption{The main screen of the Application}
			\end{figure}	
		}
		\end{frame}
	\section{Analysis of Results}
		\begin{frame}
			\frametitle{title}
		\end{frame}
	\begin{frame}
		test
	\end{frame}
\end{document}